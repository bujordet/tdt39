{\bf \Huge Plan for research studies} \\[0.5cm]
My research is divided into two phases, a research project in the fall and master thesis in the following spring. This research plan will cover both phases.
\section*{Purpose}
%"Describe the reason for doing the research, the topic of interest, why it is important or useful to study this, the specific research question(s) asked and the objectives set. Research without a purpose is unlikely to be good research." (Oates 2005)

%    Write in this section about the following:
%    •	What is the problem? Whose problem is it? Refer to 2-3                good scientific references that confirm for the reader                that this is actually a relevant problem.
%    •	What is done earlier to address this problem? Give 4-5                good references that illustrate the different approaches              taken by other researchers to solve this problem. (you                might also consider to do an in-depth literature review               as part of your autumn project).
%   •	What is wrong with earlier research? Did they not manage              to solve the problem? Why? Why is your approach going to              be better? What new knowledge do you plan to add?
%    •	Summarize with a list of clearly defined research                     questions. One main question and 2-3 sub-questions                    related to the main question. Note that the sub-questions             should contribute to answer the main question.
The Construction Industry (CI) has been a significant part of engineering throughout history. Over the past century, the requirements of constructions have become more and more complex \cite{wood2009factors}. The buildings are getting higher, the tunnels are getting longer, and the roads are getting wider. Sure, the size of things is not equal to the complexity of the construction; however, when considering automated systems, multipurpose functionality, and multiple communication platforms – the complexity is increasing. The increased complexity leads to a significant decline in labor productivity (LP), seen over the past two centuries, mentioned in the article written by SSB \cite{productivity}. As well, managing these projects is much more intricate then it used to, because of the increased numbers of actors participating in the project. 

The challenges the CI is experiencing, as well as the process used, are highly similar to what the ICT-industry was facing in the late '80s. The ICT-industry, using the waterfall process \cite{royce}, often faced the challenge of meeting the budgets and timelines. This breach had the origin in change of requirements during production, challenges in testing, and resultingly failing to deliver a finished product without bugs \cite{boehm1987software}. These problems have been frequently present when creating large and highly user-interactive software — making for the introduction of agile software development to manage theses problems. Over the years, most of the process- and method-management in ICT is digitilized. Giving tools in which both the software developers and project managers use to aid project progression.  

Frank Garry, in 1997, first introduced 3-D modeling in CI, when constructing the Peter B. Lewis Building. 3-D modeling was used both to manage the complexity of the installation, but also led to increased cooperation between different parties within the project. The paper\cite{frank_gehry}, describing this project, is reporting a change in how actors in the construction react to using computer-aided constructions, in 3-D. Today 3-D modeling is used in almost all construction projects and is known as BiM. Even though the this project showed promising results in means of cooperation and interaction, the introduction of 3-D modeling and BiM was not a single solution to the problem.

Furthermore, one has introduced Lean in the CI. The book \cite{lean_i_praksis} describes the making of the Bergen Academy of Art and Design-building, where Lean was one of the essential strategies. The object of the Case Study in this research are using experience from this book when managing the constructions. 

The motivation for this rearch is to examine a project utilizing Lean in project management, to face the problems mentioned. Furthermore, looking at how a project make use of digital tools, aiding Lean has not been examined before. Taking experience from the ICT-industry, and the use of computer-aided agile development management is also desirable, as well as looking at the problem from a different perspective. 

This project, therefore, aims to examine the fundamental reason for the striking difference in labor productivity between the ICT-industry and construction industry. The intention is not to measure productivity, but rather understand why the difference occurs.

{\bf RQ1:} Why does the difference in LP between the ICT-industry and CI appear? \\

{\bf RQ2:} How does the difference appear in the case object, which utilize both agile and digital tools?

\section*{Contributions}
%"Describe the outcomes of research, especially your contribution to knowledge about your subject area. Your contribution can be an answer to your original research question( s) but can also include unexpected findings. For example, you and the academic community might learn something about a particular research strategy as a result of your research. Your thesis, dissertation, conference paper or journal article is also a product of your research. For those research projects that involve design and creation, a new computer-based product or new development method could also be a product of your research." (Oates 2005)

%Types of research products: 1) new or improved evidence; 2) new or improved methodology; 3) new or improved analysis; 4) new or improved concepts or theories; 5) new or improved computer-based product.

%Write in this section about the following:
%•	What will be your research contribution? Which of the          four types above?
%•	How and why will it be better than earlier contributions?

The main contribution of the study is improved knowledge and theory of the of the fundamental reason for the striking gap in LP, between the CI and ICT. The project thesis has two valuable contributions: (1) the literature review; and (2) guidance as to where to target the research in the master thesis. The master thesis will consist of data collection and the analysis of the data, using a theme analysis, where the result of the analysis adds to the main contribution. 

There is little to none studies of this particular problem, but will contribute to the knowledge in implementation and use of software in complex organizations. The use and implementation of new software are considered a significant field of study in computer science, but using this knowledge in the CI will, consequently, give insights they never have and rigorously wants, given the will for digitalization; hence, this study is desirable.

\section*{Research Method}
%"Describe the sequence of activities undertaken in any research project. The process involves identifying one or more research topics, establishing a conceptual framework (the way you choose to think about your research topic), the selection and use of a research strategy and data generation methods, the analysis of data and the drawing of conclusions, including recognizing any limitations in your own research. As explained already, the process should be carried out systematically if the research is to be accepted as rigorous."
%Write here your (See the figure below from Oates 2005):
%•	Research strategy (survey, case study, action research,        experiment, design etc.)
%•	Is this mixed-method? Then you can have different              strategies for different steps in the design.
%•	Data generation methods (interview, focus group,               observation, documents etc.)
%•	You can have more than one data generation method for          each strategy.
%• 	Data analysis methods (qualitative or quantitative.)

\begin{wrapfigure}{l}{0.45\textwidth}
    \includegraphics[scale=0.23]{empirisk_studie.png}
    \caption{The research process used, marked with methods applied in the research.}
    \label{fig:my_label}
\end{wrapfigure}

This project originated in the interest in Software methodology, and how software aids productivity and cooperation in different parts of society. The projects formed when Patrick Stormo Hjerpseth, who is the Project Lead in Digitalization at the case object, gave a tip regarding writing a master thesis about object.

The project started with a literature review, providing a conseptual framework for the project. Based on the literature review, one could see a clear correlation between software engineering (SE) 30 years ago and the construction engineering (CE) of today. To identify how these problems can be faced in a project, using agile management methods and CSCW, the master thesis is planning a case study-strategy. The case studied is the the Life Science Building construction project. 

First, the research will do a minor empirical study in the project thesis, utilizing observation and interviews as data generators. This research will hopefully give insights that will bring the research to a narrower focus on the following master thesis. 

Secondary, in the master thesis, conduction of interviews and a questionnaire will aid the analysis as data generation methods. If necessary, observations will be conducted. The questionnaire will give questions on a standardized format that can be valuable in the analyses. Furthermore, the interviews will help to discover matters not covered in the questionnaire. Both interviews, the questionnaires, and observations will give quantitative data for the analysis.


\section*{Participants}
%"These include those whom you directly involve in your research, for example by interviewing them or observing them, and also those who are indirectly involved, such as the editors to whom you submit a research paper. It is important that you deal with all these people legally and ethically, that is, you do not do anything that might annoy them or cause them harm (physically, mentally or socially). You yourself as a researcher are also a research participant. As we shall see later, for some types of research, researchers are expected to be objective and remain largely unseen in the reporting of their research, whereas in other types of research the researchers are open about their feelings and how their presence influenced the other participants and the research situation."
%You need to clarify the following:
%•	Who are the informants? How do you plan to recruit them?
%•	If you want to collect data you should take care of the ethical       issues. E.g. 1) you should get approval from NSD before you           start collecting data, 2) you should have users sign informed         consent forms, 3) you should make sure you don't share                person-related data (e.g. you cannot have this data on Google         doc!).
%•	Don't underestimate use recruitment. It is often the part of the       research that shows to be the most problematic! Get started           early.
%•	Who are the researchers?
%•	Be specific who will do what in the project.

The project researcher, Morten Bujordet, is involved in the project, creating the plans, and conducting the research.
	 
Supervising the project is Eric Monteiro. Eric is contributing with experience in research in the implementation and use of new digital tools in large scale organizations. 

Furthermore, Statsbygg, as the construction manager, has an interest in the project: giving access to the participants in the study. With Patrick Stormo Hjerpseth as the point of contact.
	 
In the research, the actors using the digital software in the project design will be an aim for the data collection. All personal information gathered will, safely, be stored in a GDPR-compliant Cloud Service, served by NTNU. In the final report, no personal information will be published, and all actors in the data collection will be anonymized.

\section*{Research Paradigm}
%"A pattern or model or shared way of thinking. Managers sometimes talk of the need for a ‘paradigm shift’ to mean that a new way of thinking is required. In computing, we talk about programming language paradigms, for example, a group of languages that share a set of characteristics, such as the object-oriented paradigm (for example, Smalltalk and C + +). Here we are concerned with the philosophical paradigms of research. Any piece of research will have an underlying paradigm. We have noted already that different academic communities and individuals have different ideas about the kinds of research questions to ask and the process by which to answer them because they have different views about the nature of the world we live in and therefore about how we might investigate it. These different views stem from different philosophical paradigms. We shall look at three such paradigms: ‘positivism’, interpretivism’ and ‘critical research’ – each will be explained later."

The research strategy is adopting a case study-strategy, of a single case object. Data is generated using several methods: interviews, observation, and a questionnaire. This study is related to the interpretive paradigm — the thorough use of empirical observation of the participants and a desire to identify how they act on the new software used. Using interviews can lead to being subjective as all collection of data is done in interaction with the participants. This yields a qualitative collection of data. On the other hand, the survey is considered objective, with no interactions, giving a quantitive data collection, based on facts. Using both will, in the end, yield an objective study.

\section*{Final Deliverables and Dissemination}
%"The means by which the research is disseminated and explained to others. For example, it may be written up in a paper or thesis, or a conference paper is presented to an audience of conference delegates, or a computer-based product is demonstrated to clients, users or examiners. It is important that the presentation is carried out professionally – otherwise your audience might assume your whole research project was not undertaken in a professional manner."

The final deliverables of the project are three documents: (1) project thesis, (2) master thesis, and (3) a 5-7 pages report given to Statsbygg, containing the most significant findings in the research. 

\section*{References}
\renewcommand{\bibsection}{ }
\bibliographystyle{plain}
\bibliography{mybib}







